\documentclass{article}

\newcommand{\theme}{LightTheme}

\RequirePackage{etex} % for extended allocation purposes
\usepackage{amsmath,amsfonts,stmaryrd,amssymb}
\usepackage{mathtools}
\usepackage{titlesec}
\usepackage{enumitem}
\usepackage{longtable}
\usepackage{multirow}
\usepackage[ruled]{algorithm2e}
\usepackage{hyperref}
\usepackage{xcolor}
\usepackage[base]{babel}
\usepackage{lipsum}
\usepackage{graphicx}
\usepackage{systeme}
\usepackage{siunitx}
\usepackage{fixdif}
\usepackage{geometry}
\usepackage{multicol}
\usepackage{cleveref}
\usepackage{circuitikz}
\usepackage{braket}
\usepackage{verbatim}
% \usepackage{physics}

\usepackage[T1]{fontenc}
\usepackage{XCharter}
\usepackage[]{AlegreyaSans} % for sans-serif (use [sfdefault] option to set main font to sans-serif)
\usepackage[]{Alegreya} % for serif
\usepackage[scale=0.8]{plex-mono} % for monospace

\usepackage{luacode}

\begin{luacode*}
    function get_dirname()
        local dirname = os.env.PWD:match('([^/]+)$')
        return dirname
    end
    function get_default_name(name)
        local dirname = get_dirname()
        return name
    end

    function get_env(var)
        local env = os.env[var]
        return env
    end

    function include_partial(filename, delimiter)
        local contents = io.open(filename, "r"):read("*all")
        local lines = contents:gmatch("([^\r\n]*)\r?\n?")

        local add = false
        for line in lines do
            if line:match("%<*" .. delimiter .. ">") then
                add = true
            end

            if add then
                tex.print(line)
            end

            if line:match("%</" .. delimiter .. ">") then
                add = false
            end
        end
    end
\end{luacode*}

\newcommand{\includepartial}[2]{
    \directlua{include_partial("#1", "#2")}
}

\ifcsname theme\endcsname
\else
    \directlua{
        local theme = 'LightTheme'
        if get_env('TBOX_THEME') == 'DarkTheme' then
            theme = 'DarkTheme'
        end
        tex.print('\\newcommand{\\theme}{' .. theme .. '}')
    }
\fi
\directlua{
    local theme = [[\theme]]
    local catppuccin_theme = theme == 'DarkTheme' and 'mocha' or 'latte'
    tex.print('\\usepackage[' .. catppuccin_theme .. ']{catppuccinpalette}')
}
\usepackage{fontspec}

% TODO: Uncomment, download, and install fonts as needed
% I use Alegreya and plex-mono as my main fonts, so I don't really need these anymore.

% \newfontfamily{\vollkorn}{Vollkorn}[
%     Extension=.otf,
%     Ligatures = TeX,
%     UprightFont = *-Regular,
%     BoldFont = *-Bold,
%     ItalicFont = *-Italic,
%     BoldItalicFont = *-BoldItalic,
%     FontFace={mb}{n}{*-Medium},
%     FontFace={mb}{it}{*-MediumItalic},
%     FontFace={k}{n}{*-Black},
%     FontFace={k}{it}{*-BlackItalic},
% ]

% \newfontfamily{\jetbrains}{JetBrainsMonoNerdFontMono}[
%     Extension=.ttf,
%     Ligatures = TeX,
%     Contextuals=Alternate,
%     UprightFont = *-Regular,
%     BoldFont = *-Bold,
%     ItalicFont = *-Italic,
%     BoldItalicFont = *-BoldItalic,
%     FontFace={mb}{n}{*-Medium},
%     FontFace={mb}{it}{*-MediumItalic},
%     HyphenChar=None,
% ]

\usepackage[most]{tcolorbox}
\usepackage{listings}

\usetikzlibrary{arrows,calc,decorations.pathmorphing}
\tcbuselibrary{theorems,skins}

\newtcolorbox[auto counter]{boxcontainer}[2][]{
	enhanced,
	frame hidden,
	attach boxed title to top center={yshift=-0.5\baselineskip},
	title=#2,
	fonttitle={\vollkorn\it\bfseries},
	coltitle=TextColor,
	coltext=TextColor,
	borderline={1pt}{0pt}{TextColor},
	opacityback=0,
	boxed title style={
			frame hidden,
			borderline={1pt}{0pt}{TextColor},
			colback=PageColor,
		},
	#1
}

\newtcbtheorem[number within=section]{theorembox}{Theorem}{
	enhanced,
	detach title,
	sharp corners,
	frame hidden,
	boxrule=0pt,
	before upper=\tcbtitle\setlength{\parskip}{5pt}\par,
	fonttitle={\sffamily\bfseries},
	coltitle=ThmTitleColor,
	colback=ThmColor,
	coltext=TextColor,
	borderline west={3pt}{0pt}{ThmTitleColor},
	parbox=false,
}
{th}

\newtcbtheorem[number within=section]{proofbox}{Proof}{
	enhanced,
	detach title,
	frame hidden,
	attach title to upper=\quad,
	fonttitle={\sffamily\bfseries},
	coltitle=ThmTitleColor,
	coltext=TextColor,
	borderline={1pt}{0pt}{ThmTitleColor},
	opacityback=0,
	parbox=false,
}
{pf}

\newtcbtheorem[number within=section]{definitionbox}{Definition}{
	enhanced,
	detach title,
	sharp corners,
	frame hidden,
	boxrule=0pt,
	before upper=\tcbtitle\setlength{\parskip}{5pt}\par,
	fonttitle={\sffamily\bfseries},
	coltitle=DefTitleColor,
	colback=DefColor,
	coltext=TextColor,
	borderline west={3pt}{0pt}{DefTitleColor},
	parbox=false,
}
{def}

\newtcbtheorem[number within=section]{remarkbox}{Remark}{
	enhanced,
	detach title,
	sharp corners,
	frame hidden,
	boxrule=0pt,
	before upper=\tcbtitle\setlength{\parskip}{5pt}\par,
	fonttitle={\sffamily\bfseries},
	description font={\bfseries},
	coltitle=RemTitleColor,
	colback=RemColor,
	coltext=TextColor,
	borderline west={3pt}{0pt}{RemTitleColor},
	parbox=false,
}
{rem}

\newtcbtheorem[number within=section]{claimbox}{Claim}{
	enhanced,
	detach title,
	sharp corners,
	frame hidden,
	boxrule=0pt,
	before upper=\tcbtitle\setlength{\parskip}{5pt}\par,
	fonttitle={\sffamily\bfseries},
	description font={\bfseries},
	coltitle=ClaimTitleColor,
	colback=ClaimColor,
	coltext=TextColor,
	borderline west={3pt}{0pt}{ClaimTitleColor},
	parbox=false,
}
{claim}

\newtcolorbox[auto counter]{question}[2][]{
	enhanced,
	frame hidden,
	attach boxed title to top left={yshift=-0.5\baselineskip,xshift=+1em},
	title={Question #2},
	coltitle=TextColor,
	coltext=TextColor,
	borderline={1pt}{0pt}{TextColor},
	opacityback=0,
	boxrule=0.5pt,
	arc=3mm,
	breakable,
	parbox=false,
	boxed title style={
			colframe=TextColor,
			colback=PageColor,
			borderline={1pt}{0pt}{TextColor},
			arc=2mm,
		},
	#1
}

\newtcolorbox[auto counter]{solution}[2][]{
	enhanced,
	frame hidden,
	attach boxed title to top right={yshift=-0.5\baselineskip,xshift=-1em},
	title={Solution #2},
	coltitle=TextColor,
	coltext=TextColor,
	borderline={1pt}{0pt}{TextColor},
	opacityback=0,
	boxrule=0.5pt,
	arc=3mm,
	breakable,
	parbox=false,
	boxed title style={
			colframe=TextColor,
			colback=PageColor,
			borderline={1pt}{0pt}{TextColor},
			arc=2mm,
		},
	#1
}

\newtcolorbox[auto counter]{warn}[2][]{
	enhanced,
	breakable,
	frame hidden,
	sharp corners,
	detach title,
	opacityback=0,
	title={\textbf{#2}},
	coltitle=WarnColor,
	coltext=TextColor,
	before upper={%
			\tcbtitle\par
		},
	underlay={%
			\coordinate (center) at ($(interior.north west) + (0pt,-12pt)$);
			\begin{scope}
				\draw[color=WarnColor,very thick] (center) circle (5pt);
				\node at (center){\color{WarnColor}\scriptsize\bf ?};
				\draw[line width=1.5pt,color=WarnColor]
				([shift={(0pt,-20pt)}]interior.north west)
				--([shift={(0pt,5pt)}]interior.south west);
			\end{scope}
		},
	#1
}

\newtcolorbox[auto counter]{info}[2][]{
	enhanced,
	breakable,
	frame hidden,
	sharp corners,
	detach title,
	opacityback=0,
	title={\textbf{#2}},
	coltitle=InfoColor,
	coltext=TextColor,
	before upper={%
			\tcbtitle\par
		},
	underlay={%
			\coordinate (center) at ($(interior.north west) + (0pt,-12pt)$);
			\begin{scope}
				\draw[color=InfoColor,very thick] (center) circle (5pt);
				\node at (center){\color{InfoColor}\scriptsize\bf i};
				\draw[line width=1.5pt,color=InfoColor]
				([shift={(0pt,-20pt)}]interior.north west)
				--([shift={(0pt,5pt)}]interior.south west);
			\end{scope}
		},
	#1
}

\usepackage{etoolbox}

\newbool{shouldNewPage}\setbool{shouldNewPage}{true}

\newcommand{\newprob}[3][]{
    \begin{prob}{#1}
        #2
        \tcblower
        #3
    \end{prob}

    \ifbool{shouldNewPage}{
        \newpage
    }
}

\newcounter{Theorems}
\newenvironment{theorem}[2][]{
    \stepcounter{Theorems}
    \begin{theorembox}[#1]{#2}{\theTheorems}
        }{
    \end{theorembox}
}

\newenvironment{proof}[2][]{
    \begin{proofbox*}[#1]{#2}{}
        }{
    \end{proofbox*}
}

\newcounter{Definitions}
\newenvironment{definition}[2][]{
    \stepcounter{Definitions}
    \begin{definitionbox}[#1]{#2}{\theDefinitions}
        }{
    \end{definitionbox}
}

\newcounter{Remarks}
\newenvironment{remark}[2][]{
    \stepcounter{Remarks}
    \begin{remarkbox}[#1]{#2}{\theRemarks}
        }{
    \end{remarkbox}
}

\newenvironment{claim}[2][]{
    \begin{claimbox*}[#1]{#2}{}
        }{
    \end{claimbox*}
}

\newcounter{Propositions}
\newenvironment{prop}[2][]{
    \stepcounter{Propositions}
    \begin{propbox}[#1]{#2}{\thePropositions}
        }{
    \end{propbox}
}

\newcounter{Lemmas}
\newenvironment{lemma}[2][]{
    \begin{lemmabox}[#1]{#2}{\theLemmas}
        }{
    \end{lemmabox}
}

\newcounter{Corollaries}
\newenvironment{corollary}[2][]{
    \begin{corollarybox}[#1]{#2}{\theCorollaries}
        }{
    \end{corollarybox}
}

\newcounter{Example}
\newenvironment{example}[2][]{
    \stepcounter{Example}
    \begin{examplebox}[#1]{#2}{\theExample}
        }{
    \end{examplebox}
}

\usepackage[mocha]{catppuccinpalette}
\usepackage[most]{tcolorbox}

\lstdefinestyle{codelistingstyle}{
	breaklines=true,
	showstringspaces=false,
	breakatwhitespace=true,
	stringstyle = {\color{CtpGreen}},
	commentstyle={\color{CtpOverlay1}},
	basicstyle = {\small\color{CtpText}\ttfamily},
	keywordstyle = {\color{CtpMauve}},
	keywordstyle = [2]{\color{CtpBlue}},
	keywordstyle = [3]{\color{CtpYellow}},
	keywordstyle = [4]{\color{CtpLavender}},
	keywordstyle = [5]{\color{CtpPeach}},
	keywordstyle = [6]{\color{CtpTeal}},
	otherkeywords = {<, >, ||, =, ?},
	morekeywords = [2]{new, create, present, email, description, creator, protect_from_forgery, before_action},
	morekeywords = [3]{PageController, ApplicationController, Page},
	morekeywords = [4]{@page},
	morekeywords = [5]{exception, do_some_for_pages, @page, @admin},
	morekeywords = [6]{<, >, ||, =, ?},
	gobble=4
}

\newtcblisting{codelisting}[1]{%
	enhanced,
	listing engine=listings,
	arc=3mm,
	boxrule=0.5mm,
	colback={CtpBase},
	listing only,
	listing options={
			language={#1},
			style={codelistingstyle}
		},
	overlay={
			% Positioning the language label in the top right corner
			\node[
				anchor=north east,
				font=\small\ttfamily\bfseries,
				text=white,
				draw=CtpOverlay1,
				fill=CtpMantle,
				inner sep=2pt,
				rounded corners=3pt,
				line width=0.3mm,
				xshift=-1.5mm,
				yshift=-1.0mm
			]
			at (frame.north east) {#1};
		}
}

\usepackage{amsmath}
\usepackage{amssymb}
\usepackage{xparse} % for flexible argument parsing

\DeclareMathOperator{\proj}{proj}
\DeclareMathOperator{\Span}{Span}
\DeclareMathOperator{\Nul}{Nul}
\DeclareMathOperator{\Col}{Col}
\DeclareMathOperator{\Row}{Row}
\DeclareMathOperator{\Tr}{Tr}
\DeclareMathOperator{\rank}{rank}
\DeclareMathOperator{\nulty}{nullity}
\DeclareMathOperator{\dist}{dist}
\DeclareMathOperator{\mult}{mult}

\DeclareMathOperator{\curl}{curl}
\DeclareMathOperator{\divg}{div} % \div is in use

\DeclareMathOperator{\CRe}{Re}
\DeclareMathOperator{\CIm}{Im}

\DeclareMathOperator{\R}{\mathbb{R}}
\DeclareMathOperator{\N}{\mathbb{N}}
\DeclareMathOperator{\Z}{\mathbb{Z}}
\DeclareMathOperator{\Q}{\mathbb{Q}}
\DeclareMathOperator{\C}{\mathbb{C}}
\DeclareMathOperator{\F}{\mathbb{F}}

\DeclareMathOperator{\Rel}{Rel}

\DeclareMathOperator{\Var}{Var}
\DeclareMathOperator{\Cov}{Cov}
\DeclareMathOperator{\supp}{supp}
\DeclareMathOperator{\SD}{SD}
\DeclareMathOperator{\Bern}{Bern}         % Bernoulli Random Variable
\DeclareMathOperator{\Bin}{Bin}           % Binomial Random Variable
\DeclareMathOperator{\Poisson}{Poisson}   % Poisson Random Variable
\DeclareMathOperator{\Geom}{Geom}         % Geometric Random Variable
\DeclareMathOperator{\Uniform}{Uniform}   % Uniform Random Variable
\DeclareMathOperator{\Normal}{Normal}     % Normal Random Variable
\DeclareMathOperator{\Exp}{Exp}           % Exponential

\DeclareRobustCommand{\uvec}[1]{\hat{\mathbf{#1}}}

% \newenvironment{amatrix}[1]{%
%     \left(\begin{array}{@{}*{#1}{c}|c@{}}
%         }{%
%     \end{array}\right)
% }

% #1 = columns before bar (mandatory)
% #2 = columns after bar (optional, defaults to 1)
\NewDocumentEnvironment{amatrix}{m o}{%
\IfNoValueTF{#2}
{\left(\begin{array}{@{}*{#1}{c}|c@{}}}
{\left(\begin{array}{@{}*{#1}{c}|*{#2}{c}@{}}}
        }{%
    \end{array}\right)
}

\usepackage{pgfopts}
\usepackage{xcolor}
\usepackage{pgfplots}

\preparecolorset{HTML}{DarkTheme}{}{%
    PageColor,1E1E2E;%
    TextColor,CDD6F4;%
    ThmColor,473a59;%
    ThmTitleColor,cba6f7;%
    CorColor,5E5854;%
    CorTitleColor,f9e2af;%
    DefColor,292D42;%
    DefTitleColor,87B0F9;%
    RemColor,342D37;%
    RemTitleColor,FAB387;%
    ClaimColor,2C3239;%
    ClaimTitleColor,A6E3A1;%
    PropColor,493948;%
    PropTitleColor,eba0ac;%
    LemmaColor,5E5854;%
    LemmaTitleColor,f9e2af;%
    ExampleColor,2C3239;%
    ExampleTitleColor,A6E3A1;%
    ProblemColor,1E1E2E;%
    ProblemTitleColor,CDD6F4;%
    InfoColor,87B0F9;%
    WarnColor,FAB387;%
    HyperlinkColor,B1D1E1%
}

\preparecolorset{HTML}{LightTheme}{}{%
    PageColor,FFFFFF;%
    TextColor,000000;%
    ThmColor,D1C8DB;%
    ThmTitleColor,5C3981;%
    CorColor,D1C8DB;%
    CorTitleColor,5C3981;%
    DefColor,B1D1E1;%
    DefTitleColor,053a56;%
    RemColor,f9cab4;%
    RemTitleColor,983B0F;%
    InfoColor,053a56;%
    ClaimColor,b1e6ac;%
    ClaimTitleColor,0b8000;%
    PropColor,ffce9d;%
    PropTitleColor,ad5600;%
    LemmaColor,b1e6ac;%
    LemmaTitleColor,0b8000;%
    ExampleColor,b1e6ac;%
    ExampleTitleColor,0b8000;%
    ProblemColor,FFFFFF;%
    ProblemTitleColor,000000;%
    WarnColor,ff4c00;%
    HyperlinkColor,17a1e6%
}

\preparecolorset{HTML}{SolarizedLightTheme}{}{%
    PageColor,fdf6e3;%
    TextColor,657b83;%
    ThmColor,CDCAD9;%
    ThmTitleColor,6c71c4;%
    CorColor,CDCAD9;%
    CorTitleColor,6c71c4;%
    DefColor,C7DBDF;%
    DefTitleColor,268bd2;%
    RemColor,F8D5C5;%
    RemTitleColor,dc322f;%
    InfoColor,268bd2;%
    ClaimColor,D5D797;%
    ClaimTitleColor,859900;%
    PropColor,D5D797;%
    PropTitleColor,859900;%
    LemmaColor,D5D797;%
    LemmaTitleColor,859900;%
    ExampleColor,D5D797;%
    ExampleTitleColor,859900;%
    ProblemColor,fdf6e3;%
    ProblemTitleColor,657b83;%
    WarnColor,dc322f;%
    HyperlinkColor,17a1e6%
}

\colorlet{PageColor}{\theme PageColor}
\colorlet{TextColor}{\theme TextColor}
\colorlet{ThmColor}{\theme ThmColor}
\colorlet{ThmTitleColor}{\theme ThmTitleColor}
\colorlet{CorColor}{\theme CorColor}
\colorlet{CorTitleColor}{\theme CorTitleColor}
\colorlet{DefColor}{\theme DefColor}
\colorlet{DefTitleColor}{\theme DefTitleColor}
\colorlet{RemColor}{\theme RemColor}
\colorlet{RemTitleColor}{\theme RemTitleColor}
\colorlet{ClaimColor}{\theme ClaimColor}
\colorlet{ClaimTitleColor}{\theme ClaimTitleColor}
\colorlet{PropColor}{\theme PropColor}
\colorlet{PropTitleColor}{\theme PropTitleColor}
\colorlet{LemmaColor}{\theme LemmaColor}
\colorlet{LemmaTitleColor}{\theme LemmaTitleColor}
\colorlet{ExampleColor}{\theme ExampleColor}
\colorlet{ExampleTitleColor}{\theme ExampleTitleColor}
\colorlet{ProblemColor}{\theme ProblemColor}
\colorlet{ProblemTitleColor}{\theme ProblemTitleColor}
\colorlet{InfoColor}{\theme InfoColor}
\colorlet{WarnColor}{\theme WarnColor}
\colorlet{HyperlinkColor}{\theme HyperlinkColor}

\pgfplotsset{
    width=7cm,
    compat=1.18,
    colormap={standard-dark-theme}{
            HTML(0cm)=(cba6f7);
            HTML(1cm)=(87B0F9);
        },
    colormap={standard-light-theme}{
            HTML(0cm)=(87B0F9);
            HTML(1cm)=(cba6f7);
        },
    every axis legend/.append style = {
            at={(0,0)},
            anchor=south west,
            fill=PageColor,
            draw=TextColor
        }
}



\sisetup{
    input-ignore={.},
    group-separator = {.},
    input-decimal-markers={,},
    output-decimal-marker = {,},
    group-minimum-digits=4
}
\lstset{
    basicstyle=\ttfamily,
}
\hypersetup{
    colorlinks=true,
    pdfborder={0 0 0},
    filecolor=HyperlinkColor,
    linkcolor=HyperlinkColor,
    urlcolor=HyperlinkColor,
}
\geometry{
    paper=a4paper,
    top=2.5cm,
    bottom=3cm,
    left=2.5cm,
    right=2.5cm,
    headheight=14pt,
    footskip=1.5cm,
    headsep=1.2cm,
}

\pagecolor{PageColor}
\color{TextColor}

\setbool{shouldNewPage}{true}
\setlength{\parindent}{0pt}
\setlength{\parskip}{1em}

% TODO: Update author info, and date format!
\author{Yunus \\ \texttt{\href{github.com/yunusey}{\textcolor{TextColor}{@yunusey}}}}
\date{\today}


\title{Example Document}

\begin{document}

\maketitle

\section{Let's have some fun!}

\begin{theorem}{Stokes' Theorem}
    For a surface $\mathcal{S}$ defined on a region $\mathcal{D}$ with a boundary $\mathcal{C}$, in a vector field $\vec{\mathbf{F}}$, Stokes' theorem states that:
    \begin{equation}
        \begin{split}
            \oint_\mathcal{C} \vec{\mathbf{F}} \cdot \d \mathbf{r} & = \iint_\mathcal{D} (\nabla \times \vec{\mathbf{F}}) \cdot \mathbf{n} \d \sigma \\
        \end{split}
    \end{equation}
\end{theorem}

\begin{remark}{Green's Theorem \& Stokes' Theorem}
    Essentially, Green's theorem is a special case of Stokes' theorem. If $\mathcal{C}$ is a curve in the $xy$-plane, oriented clockwise, and $\mathcal{D}$ is the region in the $xy$-plane bounded by $\mathcal{C}$, then $\d \sigma = \d x \d y$ and $\mathbf{n} = \mathbf{k}$ and therefore,
    \begin{equation}
        \begin{split}
            (\nabla \times \vec{\mathbf{F}}) \cdot \mathbf{n}
             & = (\nabla \times \vec{\mathbf{F}}) \cdot (\mathbf{k})                          \\
             & = \left( \frac{\partial N}{\partial x} - \frac{\partial M}{\partial y} \right)
        \end{split}
    \end{equation}
\end{remark}

\begin{boxcontainer}{}
    \centering
    \begin{tikzpicture}
        \begin{axis}[
                title={Surface $\mathcal{S}$, its Boundary Curve $\mathcal{C}$, and Vector Field $\vec{\mathbf{F}}$},
                colorbar left,
                colormap name=viridis,
                xlabel=$x$, ylabel=$y$,
                variable=\u, variable y = \v,
                domain=0:360, y domain=0:3,
                legend pos=outer north east,
            ]
            \addplot3[
                surf,
                mesh/ordering=y varies,
            ] ( {v * cos(u)}, {v * sin(u)}, {v * v} );
            \addplot3[
                orange,
                mesh/ordering=y varies,
                -stealth,
                samples=20,
                quiver={
                        u=-x,
                        v=cos(y),
                        w=0,
                        scale arrows=0.2
                    },
            ] ( {v * cos(u)}, {v * sin(u)}, {v * v} );
            \addplot3[
                red,
                ultra thick,
                label={\mathcal{C}},
                domain=0:360,
                samples=60,
                samples y = 0,
            ] ( {3 * cos(u)}, {3 * sin(u)}, {9} );
            \legend{$\mathcal{S}$, $\vec{\mathbf{F}}$, $\mathcal{C}$}
        \end{axis}
    \end{tikzpicture}
\end{boxcontainer}

\begin{info}{Interesting Fact}
    Stokes' theorem is named after Sir George Gabriel Stokes, who formulated it in the 19th century. It has applications in various fields, including fluid dynamics, electromagnetism, and differential geometry.
\end{info}
\begin{warn}{Caution}
    When applying Stokes' theorem, ensure that the surface $\mathcal{S}$ is smooth and oriented correctly with respect to its boundary curve $\mathcal{C}$. The orientation of $\mathcal{C}$ must be consistent with the right-hand rule applied to the normal vector $\mathbf{n}$ of the surface.
\end{warn}

We might as well do some linear algebra while we're at it!
\begin{theorem}{}
    Let $\lambda_1, \lambda_2, \ldots, \lambda_k$ be distinct eigenvalues. Assume \[
        \begin{cases}
            S_1 & = \{ v_1^{(1)}, v_2^{(1)}, \ldots, v_{d_1}^{(1)} \} \\
            S_2 & = \{ v_1^{(2)}, v_2^{(2)}, \ldots, v_{d_2}^{(2)} \} \\
            \vdots                                                    \\
            S_k & = \{ v_1^{(k)}, v_2^{(k)}, \ldots, v_{d_k}^{(k)} \} \\
        \end{cases}
    \]
    where $S_i$ is a linearly independent set of eigenvectors corresponding to $\lambda$. Then, \[
        S = S_1 \cup S_2 \cup \cdots \cup S_k
    \]
    is linearly independent.
\end{theorem}
\begin{proof}{}
    Since the linear combinations of the eigenvectors of an eigenspace are still in the eigenspace, we have that \[
        \begin{cases}
            v_1 = a_1^{(1)} v_1^{(1)} + a_2^{(1)} v_2^{(1)} + \cdots + a_{d_1}^{(1)} v_{d_1}^{(1)} & \in \Nul(A - \lambda_1 I) \\
            v_2 = a_1^{(2)} v_1^{(2)} + a_2^{(2)} v_2^{(2)} + \cdots + a_{d_2}^{(2)} v_{d_2}^{(2)} & \in \Nul(A - \lambda_2 I) \\
            \vdots                                                                                 & \vdots                    \\
            v_k = a_1^{(k)} v_1^{(k)} + a_2^{(k)} v_2^{(k)} + \cdots + a_{d_k}^{(k)} v_{d_k}^{(k)} & \in \Nul(A - \lambda_k I) \\
        \end{cases}
    \]
    Then, we want to show that the only solution to \[
        v_1 + v_2 + \cdots + v_k = 0
    \]
    is the trivial solution $v_1 = v_2 = \cdots = v_k = 0$.  Applying $A$ to both sides, we have that \[
        A(v_1 + v_2 + \cdots + v_k) = \lambda_1 v_1 + \lambda_2 v_2 + \cdots \lambda_k v_k = A0 = 0.
    \]
    So, we want to solve the system (with $\lambda_i \neq \lambda_j$), \[
        \begin{cases}
            v_1 + v_2 + \cdots + v_k                               & = 0 \\
            \lambda_1 v_1 + \lambda_2 v_2 + \cdots + \lambda_k v_k & = 0 \\
        \end{cases}
    \]
    Inductively, we can show that this implies $v_1 = v_2 = \cdots = v_k = 0$. Thus, the set $S$ is linearly independent.
\end{proof}

Now, let's do some probability!
\begin{prop}{Properties of Independent Random Variables}
    \begin{enumerate}[label=(\alph*)]
        \item Let $X_1, X_2$ be independent random variables. Let $u_1, u_2 : \R \to \R$ be functions. Then, \[
                  E[u_1(X_1) \cdot u_2(X_2)] = E[u_1(X_1)] \cdot E[u_2(X_2)].
              \]
        \item Let $X_1, X_2, \ldots, X_n$ be independent random variables. Then, the linear combination of these random variables, \[
                  Y = \sum_{i = 1}^{n} a_i X_i,
              \]
              satisfies \[
                  E[Y] = \sum_{i = 1}^{n} a_i E[X_i], \quad \Var(Y) = \sum_{i = 1}^{n} a_i^2 \Var(X_i).
              \]
    \end{enumerate}
\end{prop}

\begin{prop}{}
    If $X \sim \Gamma(s, \lambda)$ and $Y \sim \Gamma(t, \lambda)$ are independent, then $X + Y \sim \Gamma(s + t, \lambda)$.
\end{prop}
\begin{proof}{}
    We know that \[
        f_X(x) = \frac{\lambda e^{-\lambda x} (\lambda x)^{s - 1}}{\Gamma(s)}, \quad 0 < x < \infty,
    \]
    and \[
        f_Y(y) = \frac{\lambda e^{-\lambda y} (\lambda y)^{t - 1}}{\Gamma(t)}, \quad 0 < y < \infty.
    \]
    Now, we will find the PDF of $X + Y$ for $0 < x < \infty$ and $0 < y < \infty$:
    \begin{equation}
        \begin{split}
            f_{X + Y}(a)
             & = \int_{0}^{a} f_X(x) f_Y(a - x) \d x                                                                                                                           \\
             & = \int_{0}^{a} \frac{\lambda e^{-\lambda x} (\lambda x)^{s - 1}}{\Gamma(s)} \cdot \frac{\lambda e^{-\lambda (a - x)} (\lambda (a - x))^{t - 1}}{\Gamma(t)} \d x \\
             & = \frac{\lambda^{s + t} e^{-\lambda a}}{\Gamma(s) \Gamma(t)} \int_{0}^{a} x^{s - a} (a - x)^{t - 1} \d x                                                        \\
        \end{split}
    \end{equation}
    Now, if we do a $u$-substitution with $u = \frac{x}{a}$, then $\d x = a \d u$. Thus,
    \begin{equation}
        \begin{split}
            f_{X + Y}(a)
             & = \frac{\lambda^{s + t} e^{-\lambda a}}{\Gamma(s) \Gamma(t)} \int_{0}^{1} (au)^{s - 1} (a - au)^{t - 1} a \d u         \\
             & = \frac{\lambda^{s + t} e^{-\lambda a}}{\Gamma(s) \Gamma(t)} a^{s + t - 1} \int_{0}^{1} u^{s - 1} (1 - u)^{t - 1} \d u \\
             & = C e^{-\lambda a} a^{s + t - 1}
        \end{split}
    \end{equation}
    Now, using the fact that \[
        \int_{-\infty}^{\infty} f_{X + Y}(a) \d a = 1,
    \]
    we can solve for $C$:
    \begin{equation}
        \begin{split}
            1 & = C \int_{0}^{\infty} e^{-\lambda a} a^{s + t - 1} \d a \\
        \end{split}
    \end{equation}
    Let $u = \lambda a$. Then, $\d u = \frac{1}{\lambda} \d a$. Thus,
    \begin{equation}
        \begin{split}
            1 & = C \int_{0}^{\infty} e^{-u} \left( \frac{u}{\lambda} \right)^{s + t - 1} \frac{1}{\lambda} \d u \\
              & = C \frac{1}{\lambda^{s + t}} \int_{0}^{\infty} e^{-u} u^{s + t - 1} \d u                        \\
              & = C \frac{\Gamma(s + t)}{\lambda^{s + t}}                                                        \\
        \end{split}
    \end{equation}
    Then, \[
        C = \frac{\lambda^{s + t}}{\Gamma(s + t)}.
    \]
    So, the PDF of $X + Y$ is given by \[
        f_{X + Y}(a) = \frac{\lambda^{s + t} e^{-\lambda a} a^{s + t - 1}}{\Gamma(s + t)}, \quad 0 < a < \infty.
    \]
    This is exactly the PDF of $\Gamma(s + t, \lambda)$.
\end{proof}

\begin{corollary}{}
    If $X_1, X_2, \ldots, X_n \sim \Exp(\lambda)$ are independent, then $X_1 + X_2 + \cdots + X_n \sim \Gamma(n, \lambda)$.
\end{corollary}

Let's finish off with a Numerical Analysis problem.
\newprob{
    Suppose we are given $n$ data points $(x_1, y_1), (x_2, y_2), \ldots, (x_n, y_n)$, and we want to fit them with the model \[
        f(x) = a + be^{-x}.
    \]
    The goal is to determine $a$ and $b$ by minimizing the squared error \[
        E = \sum_{i = 1}^n (a + be^{-x_i} - y_i)^2.
    \]
    Rewrite this least squares fitting problem as a system of equations in the unknowns $a$ and $b$.
}{
    First of all, we need to write the normal equations by setting the gradients to zero:
    \[
        \frac{\partial E}{\partial a} = \frac{\partial}{\partial a}\left( \sum_{i = 1}^n (a + be^{-x_i} - y_i)^2 \right) = 2 \sum_{i=1}^n (a + be^{-x_i} - y_i) = 0,
    \]
    and likewise for $b$:
    \[
        \frac{\partial E}{\partial b} = \frac{\partial}{\partial b}\left( \sum_{i = 1}^n (a + be^{-x_i} - y_i)^2 \right) = 2 \sum_{i=1}^n (a + be^{-x_i} - y_i)e^{-x_i} = 0.
    \]
    From the first equation, we have \[
        \sum_{i = 1}^n a + be^{-x_i} = \sum_{i = 1}^n y_i \implies na + b\sum_{i = 1}^n e^{-x_i} = \sum_{i = 1}^n y_i.
    \]
    And similarly, from the second equation, we have \[
        \sum_{i=1}^n a e^{-x_i} + b\sum_{i=1}^n e^{-2x_i} = \sum_{i=1}^n y_i e^{-x_i}.
    \]
    Then, in the matrix form, we want to find $a$ and $b$ such that \[
        \begin{pmatrix}
            n                     & \sum_{i=1}^n e^{-x_i}  \\
            \sum_{i=1}^n e^{-x_i} & \sum_{i=1}^n e^{-2x_i} \\
        \end{pmatrix}
        \begin{pmatrix}
            a \\
            b \\
        \end{pmatrix}
        =
        \begin{pmatrix}
            \sum_{i=1}^n y_i          \\
            \sum_{i=1}^n y_i e^{-x_i} \\
        \end{pmatrix}.
    \]
    The equivalent system of equations is therefore \[
        \begin{cases}
            na + b\sum_{i = 1}^n e^{-x_i} = \sum_{i = 1}^n y_i, \\
            a\sum_{i=1}^n e^{-x_i} + b\sum_{i=1}^n e^{-2x_i} = \sum_{i=1}^n y_i e^{-x_i}.
        \end{cases}
    \]
}

\end{document}
