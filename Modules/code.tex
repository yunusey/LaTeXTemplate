\usepackage[mocha]{catppuccinpalette}
\usepackage[most]{tcolorbox}

\lstdefinestyle{codelistingstyle}{
	breaklines=true,
	showstringspaces=false,
	breakatwhitespace=true,
	stringstyle = {\color{CtpGreen}},
	commentstyle={\color{CtpOverlay1}},
	basicstyle = {\small\color{CtpText}\ttfamily},
	keywordstyle = {\color{CtpMauve}},
	keywordstyle = [2]{\color{CtpBlue}},
	keywordstyle = [3]{\color{CtpYellow}},
	keywordstyle = [4]{\color{CtpLavender}},
	keywordstyle = [5]{\color{CtpPeach}},
	keywordstyle = [6]{\color{CtpTeal}},
	otherkeywords = {<, >, ||, =, ?},
	morekeywords = [2]{new, create, present, email, description, creator, protect_from_forgery, before_action},
	morekeywords = [3]{PageController, ApplicationController, Page},
	morekeywords = [4]{@page},
	morekeywords = [5]{exception, do_some_for_pages, @page, @admin},
	morekeywords = [6]{<, >, ||, =, ?},
	gobble=4
}

\newtcblisting{codelisting}[2][]{%
	enhanced,
	listing engine=listings,
	arc=3mm,
	boxrule=0.5mm,
	colback={CtpBase},
	listing only,
	listing options={
			language={#2},
			style={codelistingstyle}
		},
	overlay={
			% Positioning the language label in the top right corner
			\node[
				anchor=north east,
				font=\small\ttfamily\bfseries,
				text=white,
				draw=CtpOverlay1,
				fill=CtpMantle,
				inner sep=2pt,
				rounded corners=3pt,
				line width=0.3mm,
				xshift=-1.5mm,
				yshift=-1.0mm
			]
			at (frame.north east) {#2};
		},
	#1
}
