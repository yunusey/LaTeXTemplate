\usepackage[most]{tcolorbox}
\usepackage{listings}

\usetikzlibrary{arrows,calc,decorations.pathmorphing}
\tcbuselibrary{theorems,skins}

\newtcolorbox[auto counter]{boxcontainer}[2][]{
    enhanced,
    frame hidden,
    attach boxed title to top center={yshift=-0.5\baselineskip},
    title=#2,
    fonttitle={\sffamily\bfseries\itshape},
    coltitle=TextColor,
    coltext=TextColor,
    borderline={1pt}{0pt}{TextColor},
    opacityback=0,
    boxed title style={
            frame hidden,
            borderline={1pt}{0pt}{TextColor},
            colback=PageColor,
        },
    #1
}

\newtcbtheorem[number within=section]{theorembox}{Theorem}{
    enhanced,
    detach title,
    sharp corners,
    frame hidden,
    boxrule=0pt,
    before upper=\tcbtitle\setlength{\parskip}{5pt}\par,
    fonttitle={\sffamily\bfseries},
    coltitle=ThmTitleColor,
    colback=ThmColor,
    coltext=TextColor,
    borderline west={3pt}{0pt}{ThmTitleColor},
    parbox=false,
}
{th}

\newtcbtheorem[number within=section]{proofbox}{Proof}{
    enhanced,
    detach title,
    frame hidden,
    breakable,
    attach title to upper=\quad,
    fonttitle={\sffamily\bfseries},
    coltitle=ThmTitleColor,
    coltext=TextColor,
    borderline={1pt}{0pt}{ThmTitleColor},
    opacityback=0,
    parbox=false,
}
{pf}

\newtcbtheorem[number within=section]{definitionbox}{Definition}{
    enhanced,
    detach title,
    sharp corners,
    frame hidden,
    boxrule=0pt,
    before upper=\tcbtitle\setlength{\parskip}{5pt}\par,
    fonttitle={\sffamily\bfseries},
    coltitle=DefTitleColor,
    colback=DefColor,
    coltext=TextColor,
    borderline west={3pt}{0pt}{DefTitleColor},
    parbox=false,
}
{def}

\newtcbtheorem[number within=section]{remarkbox}{Remark}{
    enhanced,
    detach title,
    sharp corners,
    frame hidden,
    boxrule=0pt,
    before upper=\tcbtitle\setlength{\parskip}{5pt}\par,
    fonttitle={\sffamily\bfseries},
    description font={\bfseries},
    coltitle=RemTitleColor,
    colback=RemColor,
    coltext=TextColor,
    borderline west={3pt}{0pt}{RemTitleColor},
    parbox=false,
}
{rem}

\newtcbtheorem[number within=section]{claimbox}{Claim}{
    enhanced,
    detach title,
    frame hidden,
    attach title to upper=\quad,
    fonttitle={\sffamily\bfseries},
    coltitle=ClaimTitleColor,
    coltext=TextColor,
    borderline={1pt}{0pt}{ClaimTitleColor},
    opacityback=0,
    parbox=false,
}
{claim}

\newtcbtheorem[number within=section]{corollarybox}{Corollary}{
    enhanced,
    detach title,
    sharp corners,
    frame hidden,
    boxrule=0pt,
    before upper=\tcbtitle\setlength{\parskip}{5pt}\par,
    fonttitle={\sffamily\bfseries},
    coltitle=CorTitleColor,
    colback=CorColor,
    coltext=TextColor,
    borderline west={3pt}{0pt}{CorTitleColor},
    parbox=false,
}
{cor}

\newtcbtheorem[number within=section]{propbox}{Proposition}{
    enhanced,
    detach title,
    sharp corners,
    frame hidden,
    boxrule=0pt,
    before upper=\tcbtitle\setlength{\parskip}{5pt}\par,
    fonttitle={\sffamily\bfseries},
    coltitle=PropTitleColor,
    colback=PropColor,
    coltext=TextColor,
    borderline west={3pt}{0pt}{PropTitleColor},
    parbox=false,
}
{prp}

\newtcbtheorem[number within=section]{lemmabox}{Lemma}{
    enhanced,
    detach title,
    frame hidden,
    attach title to upper=\quad,
    fonttitle={\sffamily\bfseries},
    coltitle=LemmaTitleColor,
    coltext=TextColor,
    borderline={1pt}{0pt}{LemmaTitleColor},
    opacityback=0,
    parbox=false,
}
{lemma}

\newtcbtheorem[number within=section]{examplebox}{Example}{
    enhanced,
    detach title,
    sharp corners,
    frame hidden,
    boxrule=0pt,
    breakable,
    before upper=\tcbtitle\setlength{\parskip}{5pt}\par,
    fonttitle={\sffamily\bfseries},
    description font={\bfseries},
    coltitle=ExampleTitleColor,
    colback=ExampleColor,
    coltext=TextColor,
    borderline west={3pt}{0pt}{ExampleTitleColor},
    parbox=false,
}
{ex}

\newtcolorbox[auto counter, number within=section]{prob}[2][]{
    enhanced,
    detach title,
    sharp corners,
    frame hidden,
    boxrule=0pt,
    enforce breakable,
    before upper=\tcbtitle\setlength{\parskip}{5pt}\par,
    % title={Problem~\thetcbcounter \ifstrempty{#2}{}{:\ #2}},
    title={Problem~\thetcbcounter \ifstrempty{#2}{}{\ (#2)}},
    fonttitle={\sffamily\bfseries},
    description font={\bfseries},
    coltitle=ProblemTitleColor,
    colback=ProblemColor,
    coltext=TextColor,
    borderline west={3pt}{0pt}{ProblemTitleColor},
    parbox=false,
    segmentation style={%
            draw=ProblemTitleColor,
            solid,
            decorate,
            decoration={coil, aspect = 0, segment length=8.15mm},
        },
    #1
}

% Callout boxes, kind of similar to Markdown style--they follow the same pattern.
% So, you can create as many as you want by just calling \newtcolorbox with the calloutbox style.
% See below for two examples: `warn` and `info`.
\tcbset{
    calloutbox/.style 2 args={
            enhanced,
            breakable,
            frame hidden,
            sharp corners,
            detach title,
            opacityback=0,
            fonttitle={\sffamily\bfseries},
            coltitle=#1,
            coltext=TextColor,
            before upper={%
                    \tcbtitle\par
                },
            underlay={%
                    \coordinate (center) at ($(interior.north west) + (0pt,-12pt)$);
                    \begin{scope}
                        \draw[color=#1,very thick] (center) circle (5pt);
                        \node at (center){\color{#1}\scriptsize\bf #2};
                        \draw[line width=1.5pt,color=#1]
                        ([shift={(0pt,-20pt)}]interior.north west)
                        --([shift={(0pt,5pt)}]interior.south west);
                    \end{scope}
                },
        },
}

\newtcolorbox[auto counter]{warn}[2][]{
    calloutbox={WarnColor}{?},
    title={#2},
    #1
}

\newtcolorbox[auto counter]{info}[2][]{
    calloutbox={InfoColor}{i},
    title={#2},
    #1
}
